%% ============================================================================
%%
%%  Part A / PhD Progress Report
%%
%%  Author: Jakob Lysgaard Rørsted (Mosumgaard)
%%
%%  Front page, abstract and colophon
%% ============================================================================

% ~~~~~~~~~~~~~~~~~~~~~~~~~~~~~~~~~~~~~~~~~~~~~~~~~~~~~~~~~~~~~~~~~~~~~~~~~~~~~
% The title page
% ~~~~~~~~~~~~~~~~~~~~~~~~~~~~~~~~~~~~~~~~~~~~~~~~~~~~~~~~~~~~~~~~~~~~~~~~~~~~~

% Try to get rid of the blank verso
% --> Source: tex.stackexchange.com/questions/227711
\AtBeginShipoutNext{\AtBeginShipoutNext{\AtBeginShipoutDiscard}}

% The actual front page
\begin{titlingpage}
  \newlength{\frontpagecorrection}
  \calccentering{\frontpagecorrection}
  \begin{adjustwidth*}{\frontpagecorrection-2cm}{-\frontpagecorrection-2cm}

    \centering
    \sffamily

    \vspace*{0.1cm}

    \fontsize{26pt}{29pt}\selectfont

    \projecttitle \par

    \vspace{0.8cm}

    \fontsize{18pt}{22pt}\selectfont

    Emil Lenler-Eriksen \par

    \vspace{2.7cm}

    \includegraphics[width=5cm]{front/segla1b}
%    \includegraphics[width=5cm]{figures/sac_trim}

    \vspace{2.7cm}

    PhD Progress Report

    \vspace{1.0cm}

    \fontsize{14pt}{17pt}\selectfont

    Department of Physics and Astronomy\par
    Aarhus University\par
    % Denmark

    \vspace{0.3cm}

    May 2024

  \end{adjustwidth*}
\end{titlingpage}


% ~~~~~~~~~~~~~~~~~~~~~~~~~~~~~~~~~~~~~~~~~~~~~~~~~~~~~~~~~~~~~~~~~~~~~~~~~~~~~
% The verso of the title page
% ~~~~~~~~~~~~~~~~~~~~~~~~~~~~~~~~~~~~~~~~~~~~~~~~~~~~~~~~~~~~~~~~~~~~~~~~~~~~~

% We actually want the page numbering to start at 1 at the front page in order
% to count pages for the GSST requirements !
\setcounter{page}{2}

% ~~~~~~~~
% Abstract
% ~~~~~~~~
\vspace*{0.5cm}
\section*{Abstract}
\thispagestyle{empty}

Spectroscopy of molecular ions has been carried out by physicists for centuries. However, there is a constant push towards performing these measurements at colder temperatures in order to allow for higher precision.
Most spectroscopic methods for molecules are limited to temperatures of a few Kelvin by buffer gas cooling.  However, if the molecules are trapped in a linear Paul trap, their motions can be cooled to the motional ground state by applying laser-cooling to a co-trapped atomic ion with suitable transitions.

The work presented in this report covers the basics of trapping ions in a linear Paul trap, briefly describing the trap we use here in Aarhus, as well as going over the dynamics of systems consisting of one or two ions.
In addition, there is a description of the molecular ion source, a so-called electrospray ionization (ESI) source, which we plan to use for providing the molecular ions for our experiments.
The working concept of ESI is briefly explained, while results from a characterization of one of the octopoles within the setup are shown and discussed. Notably we find that ions may be stored in the octopole for several hours, meaning that once the ESI instrument has been filled, it can be closed off and used as a sort of "secondary ion source" for experiments.

Furthermore, we discuss how to cool down a system consisting of an atomic and a molecular ion down to the motional ground state. Doppler cooling is used to cool the atomic ion down to the so-called Doppler temperature, which is 0.5 mK for Ba$^+$ ions. While the molecular ion is not directly affected by the Doppler cooling, Coulomb interaction with the cold atomic ion should transfer heat out of the molecule in a process called sympathetic cooling. When the ions approach the Doppler temperature, they start behaving like a harmonic oscillator, at which point sideband cooling can be performed to reach the motional quantum ground state.
While Doppler cooling and sideband cooling are suitable for the case of ions with similar charge-to-mass ratios, they become unfeasible (without modification) when the ratios change too much, as the motion of the ions will become uncoupled.
In order to solve this issue, we introduce theory for coupling any two motional modes of the ions by an external field, oscillating at the difference of their frequencies. This allows transfer of energy from hard-to-cool modes into easy-to-cool modes, allowing for quantum ground state cooling of all motional modes.

Finally, there is a brief section containing a conclusion on the report, and an outlook on how we aim to proceed in the next two years.

% ~~~~~~~~~~~~
% The colophon
% ~~~~~~~~~~~~

% Get font-info
\makeatletter
\edef\fontandleading{\@memptsize.0/\the\baselineskip}
\makeatother

% % Push to bottom of page and locally set indents
% \strut\vfill
% {
%   \setlength{\parindent}{0pt}
%   \addtolength{\parskip}{.6em}

%   \begin{center}
%     \bfseries\sffamily Colophon
%   \end{center}

%   \small

%   \textsl{\projecttitle}

%   \smallskip

%   PhD progress report by Emil Lenler-Eriksen.

%   The PhD project is supervised by Michael Drewsen

%   Typeset by the author using \LaTeX{} and the \textsf{memoir} document class,
%   using Linux Libertine and Linux Biolinum {\fontandleading}.

%   %Printed at Aarhus University
% }
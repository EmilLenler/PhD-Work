%% ============================================================================
%%
%%  Part A / PhD Progress Report
%%
%%  Author: Jakob Lysgaard Rørsted (Mosumgaard)
%%
%%  Future work and outlook
%% ============================================================================

\chapter{Future Work}
\label{chap:future}
The contents of the previous chapters show the fruits of the past two years of work, but of course there is still much to be done, in order to be able to start doing PRS experiments in our trap.


Since January 2023, we have no longer been able to measure any ions at the last channeltron detector in the setup (see \cref{fig:esiDrawing}). We believe this is due to a mistake in the mounting of the push/pull feed-through of the first channeltron. There is a mesh with a circular hole below the detector mounted in front of the first channeltron, such that when the detector is pulled out of the ion beam, the ions will pass through the center of this hole, and experience symmetric surroundings, so as to avoid any deflection of the ion beam.
However, measurements that have been performed show that the previous settings used for experiments would not have had the ions passing through the center of the hole in the mesh, but rather just past the edge. If even a small shift has happened it is possible that the ions are now simply hitting the mesh altogether, giving a reason for why the ions cannot reach the latter half of the experiment.

The plan is to open up the channeltron over the course of the summer, in order to fix this issue, and obtain ions at the end of the setup once more.

When we once again have ions all the way through, we are going to start mapping out how well ions are transferred from the first to the 2nd half of the setup, and try to determine optimal parameters for transferring a single ion to the end of the setup.
Once we can reliably deliver one, or a few ions at the end of the setup, we are going to mount the octopole ion guide (see \cref{fig:fullSetup}) onto the cryogenic chamber, containing the linear Paul trap. We are then going to test protocols for moving ions through the ion guide, by attempting to move a Ba$^+$ ion to the end of the ion guide, and see if it can be brought back into the trap.

If all the above things work, all that is left is to connect all the parts of the setup together, and attempt to perform recoil spectroscopy on a molecular ion, which has a similar mass-to-charge ratio to barium.
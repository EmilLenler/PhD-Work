%% ============================================================================
%%
%%  Part A / PhD Progress Report
%%
%%  Author: Jakob Lysgaard Rørsted (Mosumgaard)
%%
%%  Introduction
%% ============================================================================

\chapter{Introduction}
\label{chap:intro}
In the 1950's Wolfgang Paul invented the so-called Paul trap, which could be used to trap charged particles within a quadrupolar electromagnetic field.\textcolor{red}{CITE}. In 1989 he would go on
 to recieve the Nobel Prize in physics, alongside Hans Dehmelt, "for the development of the ion trap technique" \textcolor{red}{CITE}.
 With the many technological and scientific advancements since the Paul trap's inception, among which the laser is an especially important one,
 it is now possible to trap, and cool single ions to temperatures below 1mK \textcolor{red}{CITE Wineland}.
 Such cold ions pose many interesting possibilities for science, as they can make good candidates for atomic clocks \textcolor{red}{CLOCK},
 or the basis for quantum computers \textcolor{red}{Wineland, Cirac Zoller}.

Paul traps are also used for the study of fluorescence of molecules in the gas phase \textcolor{red}{CITE Steen?},
where pulsed lasers can be used to excite large clouds of molecular ions, whose fluorescence spectrum may then be recorded and studied.
However as most molecules lack the necessary energy level structure for laser cooling, the temperature of these experiments are limited by their cryogenic cooling environment.